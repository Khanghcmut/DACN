\section*{Tóm tắt}
\thispagestyle{empty}
\fontsize{13}{16}
\selectfont
Căng thẳng tâm lý, một rối loạn tâm lý phổ biến, có ảnh hưởng tiêu cực đáng kể đến chất lượng cuộc sống và hiệu suất công việc, đặc biệt đối với sinh viên đang chịu áp lực phải học tập tốt. Quản lý tốt vấn đề tâm lý này là điều cần thiết để giảm thiểu hậu quả tiêu cực và cho phép can thiệp kịp thời. Nghiên cứu này sử dụng dữ liệu điện thoại thông minh - một công cụ phổ biến trong cuộc sống hiện đại - để phát triển một phương pháp mạnh mẽ để giám sát và quản lý căng thẳng từ xa.

Trong nghiên cứu này, các đặc điểm thống kê dựa trên thời lượng cuộc trò chuyện và thời lượng tĩnh; vị trí liên quan đến các số liệu dựa trên thời gian (thời gian ở trường, ở nhà, đi ăn ngoài, v.v. vào buổi sáng, trưa hoặc cả ngày) thu được từ dữ liệu GPS; tình trạng di chuyển liên quan đến các tính năng thời gian (thời gian di chuyển, thời gian hoạt động giải trí, v.v. vào buổi sáng, trưa hoặc cả ngày); ước tính các trường hợp bỏ lớp bằng cách căn chỉnh lịch học với vị trí của sinh viên; thời hạn sắp tới của sinh viên; và các tính năng dựa trên thời gian như cosin của ngày trong tuần và tuần số trong học kỳ được trích xuất từ bộ dữ liệu StudentLife.

Việc phát hiện căng thẳng đã được thử nghiệm trong hai trường hợp phân loại hai lớp (căng thẳng/không căng thẳng) và phân loại ba lớp (cảm thấy hạnh phúc, căng thẳng một chút và căng thẳng). Trong cả hai trường hợp, chúng tôi đã thử nghiệm ba mô hình học máy phổ biến: SVM, XGBoost và Random Forest. Kết quả cho thấy, trong phân loại hai lớp, hiệu suất của Random Forest đạt được độ chính xác lên đến 79\% và điểm F1 macro là 63\% và nó cũng thống trị phân loại ba lớp theo độ chính xác và điểm F1 macro lần lượt là 66\% và 51\%. Hơn nữa, chúng tôi cũng đã sử dụng Shapley Additive exPlanations (SHAP) để đánh giá hiểu biết từ các tính năng được trích xuất này. Kết quả cho thấy tính năng 'tuần trong học kỳ' đặc trưng nhất cho mức độ căng thẳng của sinh viên. Cũng cần lưu ý rằng sinh viên bỏ lớp có khả năng bị căng thẳng cao hơn một chút vào ngày hôm đó và có khả năng bị căng thẳng thấp hơn vào ngày hôm sau so với các sinh viên khác. Những hiểu biết này có giá trị cho nghiên cứu trong tương lai và cung cấp các cách tiếp cận thực tế để phát hiện và quản lý căng thẳng.

\clearpage
\pagenumbering{arabic}

\section*{Abstract}
\thispagestyle{empty}
\fontsize{13}{16}
\selectfont
 Mental stress, a common psychological disorder, has a significantly negative influence on life quality and work performance, especially for students who are under pressure to perform well in school. It is essential to handle this psychological issue well in order to minimize negative consequences and enable timely intervention. This study uses smartphone data - a ubiquitous tool in modern life - to develop a powerful approach to remotely monitor and manage stress. In this research, statistical features based on conversation duration and stationary duration; location with regard to time-based metrics (time spent at school, at home, for dining out, etc. in the morning, noon, or whole day) derived from GPS data; mobility status with respect to time features (moving time, recreational activity time, etc. in the morning, noon, or whole day); estimation of skipped class instances by aligning class schedules with student locations; the upcoming deadlines of student; and time-based features like the cosine of the day of the week and week number in the semester extracted from the StudentLife dataset. The stress detection was tested in two scenarios of two-class classifications (stress/no stress) and three-class classifications (feeling happy, a little stressed, and stressed out). In both scenarios, we tested three popular machine learning models: SVM, XGBoost, and Random Forest. The result shows that, in two-class classifications, the performance of Random Forest reached up to 79\% accuracy and 63\% macro F1 score and it also dominated the three-class classification by the accuracy and macro F1 score of 66\% and 51\% respectively. Moreover, we have also employed Shapley Additive exPlanations (SHAP) to evaluate insights from these extracted features. The results revealed that the 'week in the semester' feature is most characteristic of student stress levels. It is also worth mentioning that students who skip classes have a marginally higher likelihood of experiencing stress on that day, and a lower chance of facing stress the next day compared to their counterparts. These insights are invaluable for future research and offer practical approaches to stress detection and management.

\clearpage
\pagenumbering{arabic}

\chapter{Giới thiệu}
\section{Đặt vấn đề}\label{Problems_settings}
Stress tâm lý (hay căng thẳng tâm lý) là một vấn đề phổ biến của thế kỷ 21. Theo nghiên cứu của Cơ quan An toàn và Sức khỏe Nghề nghiệp Anh Quốc, stress chiếm tới 37\% lý do cho các chứng bệnh liên quan đến công việc trong năm 2015/2016
\cite{HSE}. Hơn thế nghiên cứu của Alert cho thấy rằng cứ 3 người Mỹ thì sẽ có 1 người có sức khoẻ tâm thần loại khá hoặc tệ \cite{american_stress}. Về tác hại của căng thẳng, hiện tượng này không chỉ ảnh hưởng đến sức khỏe tinh thần mà còn gây ra nhiều vấn đề về thể chất, như tăng huyết áp, suy giảm hệ miễn dịch, hoặc kiệt quệ. Một nghiên cứu gần đây cho thấy, sinh viên đại học đang đối mặt với mức độ stress cao, dẫn đến giảm hiệu suất học tập và tăng nguy cơ bỏ học. Thêm vào đó theo nghiên cứu của Jeffrey \cite{stress_reduce_productivity}, người bị stress có xu hướng bị giảm 5 đến 12\% năng suất làm việc dẫn đến hơn 50\% giờ làm việc của thế giới đã bị mất hằng năm do các vấn đề tâm lý \cite{workday_lost}. Vì vậy việc nghiên cứu công nghệ để nhận diện stress hằng ngày là quan trọng trong cuộc sống hiện tại.

% Về nghiên cứu về stress tâm lý, một câu hỏi được đặt ra chỉnh là khi nào cần có sự nhận diện vào trạng thái tâm lý của một con người. Để giải quyết vấn đề ấy, ta cần quan tâm đến việc stress là một hiện tượng mà vừa có thể làm động lực cho con người phát triển ở giai đoạn vừa và nhẹ và sẽ gây ra các tác hại khi dần vào các giai đoạn sâu của quá trình ấy. Hơn thế việc can thiệp về tâm lý dặt ra một thách thức về sự thao túng tâm lý của con người.

Khi xét về các ưu điểm việc nhận diện được stress mang lại, các thiết bị đeo được như đồng hồ thông minh đã nổi lên như những công cụ tiềm năng chính xác để phát hiện căng thẳng, tận dụng sự tích hợp của các cảm biến có độ nhạy cao để thu thập các thông số sức khỏe toàn diện. Hơn nữa, với sự phát triển của các công cụ phân tích dữ liệu như học máy và học sâu, dữ liệu sinh trắc trở thành một nguồn thông tin mạnh mẽ để phát hiện căng thẳng chính xác.

Đặc biệt, nhiều nghiên cứu (như nghiên cứu của Pekka Siirtola \cite{PS}, của Salai \cite{stress_heartrate}) đã đạt được độ chính xác ổn trong phát hiện căng thẳng bằng các thiết bị đeo nêu trên. Tuy nhiên, một hạn chế chính nằm ở sự cần thiết phải đeo liên tục, vì dự đoán căng thẳng chính xác bị ảnh hưởng đối với những người chỉ sử dụng thiết bị gián đoạn, chẳng hạn như khi đi lại hoặc tại nơi làm việc.

Điện thoại thông minh, một thiết bị phổ biến trong cuộc sống hiện đại, có tiềm năng đóng vai trò là bộ thu dữ liệu đáng kể để theo dõi căng thẳng liên tục. Ví dụ, Martin Gjoreski và cộng sự đã phát triển một mô hình học máy có thể phát hiện một cách kín đáo mức độ căng thẳng ở sinh viên bằng cách sử dụng dữ liệu có sẵn từ cảm biến điện thoại thông minh\cite{d}. Nhóm của Elena Vildjiounaite cũng sử dụng một số mô hình để phân tích dữ liệu điện thoại và thu được kết quả tích cực\cite{e}. Tuy nhiên, việc phân tích các tính năng dựa trên vị trí và di chuyển vẫn chưa được đầu tư khám phá nhiều, chủ yếu do những thách thức trong việc dự đoán mức độ căng thẳng theo thời gian thực so với các chỉ số sinh trắc trực tiếp hơn như nhịp tim hoặc huyết áp. Nhưng, khi được xem xét như là một phần của một bộ dữ liệu GPS lớn và liên tục kéo dài nhiều tuần, các tính năng này có tiềm năng đáng kể để dự đoán chính xác mức độ căng thẳng.
 \section{Đối tượng nghiên cứu}
 Với cuộc sống xô bồ như hiện tại, con người ta dần quên đi việc chăm sóc cho bản thân. Từ đây, các vấn đề tâm lý càng trở nên trầm trọng hơn và gây ra nhiều hệ luỵ đáng tiếc. 

 Đối đối với học sinh, việc chịu áp lực học tập trong thời gian dài có thể gây hại cho sinh viên. Việc tìm hiểu và đưa ra được gợi ý về lý do stress hoặc không stress của sinh viên là một vấn đề đang được quan tâm trong hệ thống giáo dục. Việc hiểu được và nhận dạng được các yếu tố stress không chỉ giúp sinh viên có thể dùng để làm một điểm tham chiếu cho kế hoạch học tập của mình mà còn là một công cụ để cung cấp một góc nhìn mới về các hoạt động, và hành vi của sinh viên cho các người hoạt động giáo dục. Từ đó nối gần mối quan hệ và sự thấu hiểu giữa sinh viên và giảng viên.

 % Hơn thế, đối với giảng viên, đề tài này cung cấp một góc nhìn mới về các hoạt động của sinh viên, từ đó đưa ra cái nhìn tổng quát về các hoạt động của sinh viên (hoạt động cá nhân, hoạt động trên lớp) cũng như các hành vi của sinh viên (chạy deadline, cúp học). Thông qua những góc nhìn mới đó, người hoạt động giáo dục sẽ có thêm thông tin, căn cứ để hiểu hơn về sinh viên của mình cũng như là một tham chiếu để hiểu và nâng cao chất lượng cuộc sống của sinh viên.

  Do vậy, ở đề tài này đối tượng nghiên cứu tôi lựa chọn là sinh viên (cụ thể là sinh viên của trường đại học Dartmouth), đối tượng dễ bị tổn thương nhất do có một sự thay đổi lớn về cuộc sống (do sự thay đổi địa điểm sống để phù hợp với cuộc sống đại học) cũng như đối tượng cần có nhiều sự quan tâm của xã hội. 
\section{Mục tiêu đề tài}
Với các vấn đề và các nghiên cứu liên quan ở trên, bài viết này đặt ra các mục tiêu hoàn thành bốn mục tiêu được nêu ở dưới:
\begin{itemize}
    \item Ứng dụng bộ dữ liệu hiện có (StudentLife), phân tích và khai phá đặc trưng liên quan đến tín hiệu định vị vệ tinh (GPS) đến các hoạt động của sinh viên như thời gian học tập, vui chơi giải trí, và kết hợp với các dữ liệu học tập của sinh viên (thời khoá biểu, thời hạn nộp bài (hay deadline),...) tạo một bộ dữ liệu đặc trưng cho hoạt động và hành vi của sinh viên.
    \item Sử dụng các dữ liệu trích xuất được, xây dựng mô hình phân loại trạng thái căng thẳng hàng ngày trong cộng đồng sinh viên. Điều này nhằm giúp cho sinh viên có thể hiểu về trạng thái tâm lý của chính bản thân mình và có những sự điều chỉnh về hành động của sinh viên đó.
    % specific...->purpose
    \item Ứng dụng công nghệ trí tuệ nhân tạo giải thích (XAI) đưa ra được các giải thích nguyên nhân của các vấn đề tâm lý của sinh viên. Và ứng dụng XAI để đưa ra gợi ý cho các sinh viên về sự cân bằng việc học và cuộc sống.
    \item Ứng dụng XAI, lượng hoá được sự ảnh hưởng của yếu tố nghỉ học của sinh viên, từ đó mở rộng thêm về sự hiểu biết về  một hành vi đặc biệt của sinh viên và tác động của hành vi đó với sức khoẻ tinh thần của họ.
\end{itemize}
\chapter{Tổng kết}
\section{Nhận xét}
Nghiên cứu này khám phá khả năng dự đoán của các mô hình học máy tận dụng các tính năng dựa trên GPS. Kết quả của tôi tiết lộ mối liên hệ đáng kể giữa mức độ căng thẳng của học sinh và các hành vi có thể nhận biết được thông qua dữ liệu điện thoại di động. Đặc biệt, các kỹ thuật học máy, đáng chú ý là mô hình Rừng ngẫu nhiên và XGB, cho thấy triển vọng trong việc phát hiện sớm và cảnh báo các vấn đề căng thẳng ở các cá nhân. Những phát hiện này cung cấp những hiểu biết quan trọng về dự đoán sức khỏe tâm thần, nhấn mạnh tầm quan trọng của các yếu tố như chỉ số tuần, thời gian giải trí, thời gian ở nhà, và thời gian học trong việc hiểu được căng thẳng tâm lý của học sinh.
\section{Hướng phát triển}
Về hướng phát triển, tôi đề xuất các nghiên cứu tiếp theo tập trung khai phá những yếu tố khác như thời lượng sử dụng điện thoại, quãng đường di chuyển, giấc ngủ, chủng tộc, sự thay đổi văn hoá (ở những đối tượng sinh viên trrao đổi văn hoá),... để có cái nhìn tổng quát hơn về vấn đề. Ngoài ra nghiên cứu này chỉ dừng lại ở các thuật toán học máy, các nghiên cứu tiếp theo có thể ứng dụng các giải thuật học sâu để có thể tìm ra được quy luật của trạng thái tâm lý.